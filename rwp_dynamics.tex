\documentclass[12pt]{article}
\usepackage{amsfonts}
\usepackage{booktabs}
\usepackage[margin=1in]{geometry}
\usepackage{amsmath}
\usepackage{gensymb}
\usepackage{cite}
\usepackage{sectsty}
\usepackage{graphicx}
\usepackage{array}
\usepackage{caption}
\usepackage{hyperref}
\usepackage{subfig}
\usepackage{float}
\usepackage{rotating}
\usepackage{tikz}
\usepackage{fixltx2e}
\usepackage{enumitem}
\usepackage{setspace}
\usepackage[normalem]{ulem}
\newcommand{\msout}[1]{\text{\sout{\ensuremath{#1}}}}
\newcommand\numberthis{\addtocounter{equation}{1}\tag{\theequation}}

\newenvironment{conditions}
  {\par\vspace{\abovedisplayskip}\noindent\begin{tabular}{>{$}l<{$} @{${}={}$} l}}
  {\end{tabular}\par\vspace{\belowdisplayskip}}

\setlength\parindent{0pt}
\usepackage{gensymb}
\begin{document}

\section*{Reaction Wheel Pendulum Dynamics}
To obtain the equations of motion for the system, a Lagrangian approach is used.  First, the energies and virtual work of the system are computed:

\begin{align*}
	KE_{\,1} &= \frac{1}{2}m_1\left(\frac{l_1}{2}\dot{q_1}\right)^2 + \frac{1}{2}J_1\dot{q_1}^2 \\
	KE_{\,2} &= \frac{1}{2}m_2\left(l_1\dot{q_1}\right)^2 + \frac{1}{2}J_2\left(\dot{q_1}+ \dot{q_2}\right)^2\\
	\bigskip\bigskip\bigskip
	PE_{\,1} &= m_1\,g\,\frac{l_1}{2}\,\text{cos}(q_1) \\
	PE_{\,2} &= m_2\,g\,l_1\,\text{cos}(q_1) \\
	dW &= \tau_{motor}\,dq_2
\end{align*}

where
\bigskip

\begin{conditions}
KE_{\,1} & kinetic energy of the link \\
KE_{\,2} & kinetic energy of the flywheel \\
PE_{\,1} & potential energy of the link, link pivot point is at zero height\\
PE_{\,2} & potential energy of the flywheel, link pivot point is at zero height \\
m_1 & mass of the link \\
m_2 & mass of the flywheel \\
J_1 & rotational inertia of the link \\
J_2 & rotational inertia of the flywheel \\
q_1 & angle of the link, tared at the directly inverted position \\
q_2 & angle of the flywheel with respect to the link \\
l_1 & length of the link \\
g & gravitational acceleration \\


\end{conditions}


\bigskip Then, the following Lagrangians are evaluated in terms of the two generalized coordinates, $q_1$ and $q_2$:

\begin{align*}
	\frac{d}{dt} \left( \frac{\partial L}{\partial \dot{q_1}}\right) - \frac{\partial L}{\partial q_1} &= 0  \\
	\frac{d}{dt} \left( \frac{\partial L}{\partial \dot{q_2}}\right) - \frac{\partial L}{\partial q_2} &= \tau 
\end{align*}

From simplifying these equations, the (messy) equations of motion are finally derived:


\begin{align*}
	\ddot{q_1} &= \frac{(m_1 g \frac{l_1}{2} + m_2 g l_1)q_1 - \tau}{m_1 \frac{l_1^2}{4} + J_1 + m_2 l_1^2} \\
	\ddot{q_2} &= \frac{(m_1 g \frac{l_1}{2} + m_2 g l_1)q_1 - \frac{m_1\frac{l_1^2}{4} + J_1 + m_2 l_1^2 + J_2}{J_2}\tau}{-m_1 \frac{l_1^2}{4} - J_1 - m_2 l_1^2} \\
\end{align*}

Instead of clearing denominators, we'll make a few substitutions.  Letting

\begin{align*}
	a &= m_1 \frac{l_1^2}{4} + J_1 + m_2 l_1^2 + J_2, \,\text{and} \\
	b &= m_1 g \frac{l_1}{2} + m_2 g l_1, \\
\end{align*}

the equations simplify to:

\begin{align*}
	\ddot{q_1} &= \frac{b q_1 - \tau}{a-J_2} \\
	\ddot{q_2} &= \frac{b q_1 - \frac{a}{J_2} \tau}{J_2 - a}
\end{align*}

which, when written in state space form, takes on the following representation:

\begin{equation*}
\begin{pmatrix} 
\dot{q_1}  \\
\dot{q_2} \\
\ddot{q_1} \\
\ddot{q_2} \\ 
\end{pmatrix}
= 
\begin{pmatrix}
0 & 0 & 1 & 0 \\
0 & 0 & 0 & 1 \\
\frac{b}{J_2-a} & 0 & 0 & 0 \\
\frac{b}{J_2 - a} & 0 & 0 & 0
\end{pmatrix}
\begin{pmatrix} 
q_1  \\
q_2 \\
\dot{q_1} \\
\dot{q_2} \\ 
\end{pmatrix}
+ 
\begin{pmatrix}
0 \\
0 \\
\frac{-1}{a-J_2} \\
\frac{-a}{J_2(J_2 - a)}\\
\end{pmatrix}
\tau_{motor}
\end{equation*}
\bigskip

This state-space form is ultimately used in the controller simulations in Python using the
 \href{https://pypi.org/project/control/}{Control Systems Library (link)}.






\end{document}

